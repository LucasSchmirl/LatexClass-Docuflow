
\section{PDFLaTeX Font Options}

If you compile your document with PdfLaTeX, you can change the font family using standard LaTeX font packages.
The following font families are commonly available in most LaTeX distributions.\\

\begin{itemize}[noitemsep]
    \item \rm{Serif: The quick brown fox jumps over the lazy dog.}
    \item \sf{Sans Serif: The quick brown fox jumps over the lazy dog.}
    \item \tt{Typewriter: The quick brown fox jumps over the lazy dog.}
\end{itemize}
You can also use font packages to change the font style:
\begin{itemize}[noitemsep]
    \item {\usefont{T1}{phv}{m}{n} \textbf{Helvetica}: The quick brown fox jumps over the lazy dog.}
    \item {\usefont{T1}{phv}{b}{n} \textbf{Helvetica Bold}: The quick brown fox jumps over the lazy dog.}
    \item {\usefont{T1}{ppl}{m}{n} \textbf{Palatino}: The quick brown fox jumps over the lazy dog.}
    \item {\usefont{T1}{pcr}{m}{n} \textbf{Courier}: The quick brown fox jumps over the lazy dog.}
    \item {\usefont{T1}{put}{m}{n} \textbf{Utopia}: The quick brown fox jumps over the lazy dog.}
    \item {\usefont{T1}{put}{b}{n} \textbf{Utopia Bold}: The quick brown fox jumps over the lazy dog.}
    \item {\usefont{T1}{pbk}{m}{n} \textbf{Bookman}: The quick brown fox jumps over the lazy dog.}
    \item {\usefont{T1}{pag}{m}{n} \textbf{Avant Garde}: The quick brown fox jumps over the lazy dog.}
    \item {\usefont{T1}{pag}{b}{n} \textbf{Avant Garde Bold}: The quick brown fox jumps over the lazy dog.}
    \item {\usefont{T1}{pnc}{m}{n} \textbf{New Century Schoolbook}: The quick brown fox jumps over the lazy dog.}
    \item {\usefont{T1}{pzc}{m}{n} Zapf Chancery: The quick brown fox jumps over the lazy dog.}
    \item {\usefont{T1}{pbk}{b}{n} Bookman Bold: The quick brown fox jumps over the lazy dog.}
    \item {\usefont{T1}{ptm}{m}{n} \textbf{Times}: The quick brown fox jumps over the lazy dog.}
    \item {\usefont{T1}{pnc}{b}{n} \textbf{New Century Schoolbook Bold}: The quick brown fox jumps over the lazy dog.}
    \item \textbf{Default font}: The quick brown fox jumps over the lazy dog.
\end{itemize}


% -------------------------------
% PDFLaTeX Font Selection Cheat Sheet
% Usage: \usefont{T1}{family}{series}{shape}
% -------------------------------

% Family Codes (T1 encoding):
% phv  : Helvetica (sans-serif)
% ppl  : Palatino (serif)
% pcr  : Courier (monospace)
% put  : Utopia (serif, clean)
% pbk  : Bookman (serif, decorative)
% pag  : Avant Garde (sans-serif, geometric)
% pnc  : New Century Schoolbook (serif, classic)
% ptm  : Times (serif, standard)
% pzc  : Zapf Chancery (calligraphic / script)

% Series Codes:
% m    : Medium (normal)
% b    : Bold
% bx   : Bold extended (sometimes available)

% Shape Codes:
% n    : Normal / upright
% it   : Italic
% sl   : Slanted
% sc   : Small caps
% ui   : Upright italic (rare)

% Notes:
% - These fonts are standard in most TeX distributions.
% - Use within a group {...} to limit scope:
%   {\usefont{T1}{phv}{m}{n} Some text in Helvetica}
% - For pdflatex, these are the easiest fonts to switch between using \usefont.
