% DocuFlow Class v1.0 (Tutorial)
% Created by Lucas Schmirl, 2025.

\section{Kurze Einführung in die \texttt{docuflow}-Klasse}

Die \texttt{docuflow}-Klasse ist eine LaTeX-Klasse, die speziell für wissenschaftliche Dokumente, Berichte und Präsentationen entwickelt wurde. 
Sie bietet einige Komfortfunktionen und vorgefertigte Layouts für konsistente Formatierung.

\subsection{Aufruf der Klasse}

Die Klasse wird wie folgt eingebunden:

\begin{verbatim}
\documentclass[<Sprache>,<DokTyp>,<Zitationsstil>,<Schriftgröße>]{docuflow}
\end{verbatim}

\subsection{Mögliche Optionen}

\begin{itemize}
    \item \textbf{Sprache:} 
    \begin{itemize}
        \item \texttt{english} – Englische Lokalisierung (Standard)
        \item \texttt{ngerman} – Deutsche Lokalisierung
    \end{itemize}
    \item \textbf{Dokumenttyp:} 
    \begin{itemize}
        \item \texttt{article} – Standardartikel
        \item \texttt{report} – Berichtsmodus (Kapitel verfügbar)
        \item \texttt{book} – Buchmodus (Kapitel verfügbar)
    \end{itemize}
    \item \textbf{Zitationsstil:} 
    \begin{itemize}
        \item \texttt{apa7} – APA 7. Ausgabe
        \item \texttt{harvard} – Harvard-Stil
        \item \texttt{IEEE} – IEEE-Zitationsstil
    \end{itemize}
    \item \textbf{Schriftgröße:} 
    \begin{itemize}
        \item \texttt{10pt}, \texttt{11pt}, \texttt{12pt} – Standard-LaTeX-Schriftgrößen
    \end{itemize}
\end{itemize}

\subsection{Besondere Funktionen}

\begin{itemize}
    \item \textbf{Titelblatt mit Hintergrundbild:}\\
    Mit den Makros \verb|\settitlebackground{<Bildpfad>}| und 
    
    \verb|\setTitlePageContent{<Inhalt>}| kann ein individuelles Titelblatt gestaltet werden.

    \item \textbf{Automatische Überschriften für Frontmatter:}\\
    Abstract, Kurzfassung und Abkürzungsverzeichnis passen sich automatisch an den Dokumenttyp an (Artikel: \verb|\section*|, Report/Book: \verb|\chapter*|).

    \item \textbf{Bibliographie-Integration:}\\
    Biber/BibLaTeX wird direkt unterstützt. Der Stil passt sich automatisch an die Klassenoption an.

    \item \textbf{Makros für Abbildungen:}\\
    Beispiel: \verb|\DFfigure[<Breite>]{<Bild>}{<Beschriftung>}{<Label>}|
    
    \item \textbf{Deutsche Lokalisierung:}\\
    Überschriften für Abbildungsverzeichnis, Tabellenverzeichnis, Inhaltsverzeichnis und Abkürzungsverzeichnis werden automatisch gesetzt, wenn \texttt{ngerman} gewählt wird.
\end{itemize}


\clearpage

\subsection{Aufbau der Dokumentdateien}

Die \texttt{docuflow}-Klasse ist so konzipiert, dass der Inhalt modular in mehreren \texttt{.tex}-Dateien organisiert wird. Jede Datei hat einen eigenen Zweck und kann unabhängig bearbeitet werden. Danach werden die Dateien einfach in das Hauptdokument eingebunden.

\begin{itemize}
    \item \textbf{definitions.tex:}  
    Enthält alle globalen Definitionen, Farben, Schriftarten, Größen und Makros (z. B. für Abbildungen). Diese Datei wird einmalig in \texttt{main.tex} geladen.

    \item \textbf{abstract.tex:}  
    Enthält den englischen Abstract des Dokuments. Einfach den Text bearbeiten, die Überschrift wird automatisch angepasst.

    \item \textbf{kurzfassung.tex:}  
    Enthält die deutsche Kurzfassung. Diese Datei wird nur eingebunden, wenn die Klasse mit \texttt{ngerman} aufgerufen wird.

    \item \textbf{content.tex:}  
    Der Hauptinhalt des Dokuments. Alle Kapitel, Abschnitte oder sonstiger Text wird hier eingefügt.  

    \item \textbf{bibliography.bib:}  
    Enthält alle Literaturangaben im BibLaTeX-Format. 
    
    Sie wird über \verb|\addbibresource{bibliography.bib}| in der Klasse eingebunden.

    \item \textbf{PICs/\textless name\textgreater.pdf/png/jpg:}  
    Alle Grafiken, die über \verb|\DFfigure| oder manuell eingefügt werden, liegen hier.  
\end{itemize}

\noindent
Die Idee ist, dass jede Datei separat bearbeitet wird, ohne das Hauptdokument zu verändern. Das Hauptdokument \texttt{main.tex} bindet sie dann alle ein und sorgt für die korrekte Reihenfolge.


\clearpage

\subsection{Beispiel für die Verwendung}


In \verb|definitions.tex|:

\begin{verbatim}
    \settitlebackground{PICs/bg-image.pdf}
    \setTitlePageContent{
    \centering
    {\Huge\bfseries Mein Titel}\\[2ex]
    {\Large Autor Name}\\[1ex]
    {\small \today}
    }
\end{verbatim}

\noindent In \verb|main.tex|:

\begin{verbatim}
    \documentclass[ngerman,article,IEEE,12pt]{docuflow}


    \begin{document}
    \maketitle
    \DFautoSection{Abstract}
This is the abstract of the document. You can summarize your work here.
This abstract is necessary no matter if the document is written in English or German.
    \DFautoSection{Kurzfassung}
Dies ist die Kurzfassung auf Deutsch. Hier wird die Arbeit zusammengefasst.
Dies Kurzfassung ist dann notwendig wenn die Arbeit in deutscher Sprache verfasst ist.
    \pagenumbering{arabic}
    % DocuFlow Class v1.0 (content.tex)
% Created by Lucas Schmirl, 2025.

\section{Einführung (\texttt{snippets})}
Hier steht der hauptsächliche \glqq~Inhalt\grqq des Dokuments.

Verwende das figure-macro der Klasse um Bilder einzufügen:
\DFfigure[0.6\linewidth, angle=-90]{PICs/hund}{Das ist ein Hund.}{fig:smallFigure}

Abkürzungen wie diese:
\ac{CPU}, \ac{RAM}, and \ac{GPU}, 
können nach der Definition in \texttt{acronym} in \texttt{main.tex} verwendet werden.


    \subsection{Überschrift 2. Ebene}
    Zitate können mit: \parencite{knuth1984} oder~\cite{knuth1984} gesetzt werden.

    Farbige  \textcolor{FAVblue}{Texte} funktionieren so.
    \printbibliography
    \listoffigures
    \listoftables
    \section*{\acronymname}
    \begin{acronym}[TDMA]
    \acro{CPU}{Central Processing Unit}
    \acro{RAM}{Random Access Memory}
    \end{acronym}
    \end{document}
\end{verbatim}

\clearpage
\section{Kompilieren des Dokuments}\label{settingsjson}

Um das Dokument zu kompilieren, verwenden Sie einen LaTeX-Editor wie TeXstudio oder Visual Studio Code mit der LaTeX-Workshop-Erweiterung. 
Bei der Verwendung von Visual Studio Code stellen Sie sicher, dass die \texttt{settings.json} (erreichbar mit F1) Datei folgenden Block enthält:\\

Achtung, dieser Block ist nur ein Ausschnitt der gesamten \texttt{settings.json} Datei. 
Falls Sie noch keine \texttt{settings.json} Datei haben, können Sie eine neue anlegen und den gesamten Inhalt einfügen. 
Der folgende Block muss innerhalb geschweifter Klammern stehen:

% \begin{lstlisting}
%     // Latex Settings
%     "latex-workshop.latex.autoClean.run": "onBuilt",
%     "latex-workshop.latex.autoBuild.run" : "onSave",

%     // Tools
%     "latex-workshop.latex.tools": [
%         {
%          "name": "pdflatex",
%          "command": "pdflatex",
%          "args": [
%           "-shell-escape",
%           "-synctex=1",
%           "-interaction=nonstopmode",
%           "-file-line-error",
%           "-outdir=%OUTDIR%",
%           "%DOC%"
%          ],
%          "env": {}
%         },
%         {
%          "name": "lualatex",
%          "command": "lualatex",
%          "args": [
%           "-shell-escape",
%           "-synctex=1",
%           "-interaction=nonstopmode",
%           "-file-line-error",
%           "-outdir=%OUTDIR%",
%           "%DOC%"
%          ],
%          "env": {}
%         },
%         {
%             "name": "biber",
%             "command": "biber",
%             "args": [
%                 "%DOCFILE%"
%             ]
%         },
%        ],


%        // Recipes
%        "latex-workshop.latex.recipes": [
%         {
%          "name": "pdflatex", //pdflatex ➞ biber ➞ pdflatex`×2
%          "tools": [
%           "pdflatex",
%           "biber",
%           "pdflatex",
%           "pdflatex",
%          ]
%         },
%         {
%          "name": "lualatex", //lualatex ➞ biber ➞ lualatex`×2
%          "tools": [
%           "lualatex",
%           "biber",
%           "lualatex",
%           "lualatex",
%          ]
%         }
%     ],
% \end{lstlisting}