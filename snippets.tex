% DocuFlow Class v1.0 (Snippets)
% Created by Lucas Schmirl, 2025.

\section{Hier sind weitere \texttt{Snippets}}

Querverweise werden in \LaTeX{} automatisch erzeugt und verwaltet, damit sie leicht aktualisiert werden können. 
Hier wird zum Beispiel auf Abbildung~\ref{Abb:hund} verwiesen.

Abkürzungen sind im Abkürzungsverzeichnis vermerkt. Ein Beispiel einer Abkürzung im Kontext unserer Produktionslinien ist: \ac{TBD}\\

Die \glqq{}Gute-Farbe\grqq{} ist \textcolor{FAVgreen}{dieses Grün (RGB:~177,179,48)}.

\noindent Wenn ein neuer Absatz \textcolor{red}{nicht} eingerückt werden soll funktioniert das so.

\section{Überschrift Tiefe 1 (section)}
    
    \begin{figure}[!htbp]
        \centering
        \includegraphics[width=0.7\linewidth]{PICs/hund.png}
        \caption{Das ist der selbe Hund (ohne das Bild zu rotieren).}
        \label{Abb:hund}
    \end{figure}
    
    \clearpage
        \subsection{Überschrift Tiefe 2 (subsection)}
        Und hier ist ein Verweis auf Tabelle~\ref{tab:1}. Das gezeigte Tabellenformat ist nur ein Beispiel. 
        Tabellen können individuell gestaltet werden, dazu gerne auch mal dieses \hyperref{https://www.tablesgenerator.com/}{category}{name}{Tool} austesten.\\
    
        \begin{table}[!htbp]
            \centering
            \begin{tabular}{| p{0.3\linewidth} | p{0.3\linewidth} | p{0.3\linewidth} |}\hline
            Datum & Produktionsschritt & Abteilung\\\hline
            15.02.2025 & Rohstoffmischung & Chemielabor\\
            17.02.2025 & Qualitätsprüfung & Labor\\
            20.02.2025 & Abfüllung & Produktion\\
            22.02.2025 & Verpackung & Logistik\\\hline
            \end{tabular}
            \caption{Produktionsplan für Reinigungsmittel \glqq{} EcoClean\grqq{}.}\label{tab:1}
        \end{table}



        
        \subsubsection{Überschrift Tiefe 3 (subsubsection)}
        Anführungszeichen können auf \glq{}diese\grq{} und \glqq{}jene\grqq{} Weise verwendet werden.\\

        So macht man einen Absatzumbruch.\\
        Und so einen Seitenumbruch.\clearpage


\section{Überschrift der Tiefe 0 (section)}
Im nächsten Unterkapitel werden Formeln dargestellt.


    \subsection{Überschrift Tiefe 1 (subsection)}
    Hier wird auf die Formel~\ref{Gl:1} verwiesen.

    \begin{align}
        x = -\frac{p}{2}\pm\sqrt{\frac{p^2}{4}-q}\label{Gl:1}
    \end{align}
    \begin{align}
        x = -\frac{p}{2}\pm\sqrt{\frac{p^2}{4}-q}\label{Gl:2}
    \end{align}

        \subsubsection{Überschrift Tiefe 2 (subsubsection)}
        Literaturverweise sollten automatisch verwaltet werden, vor allem, wenn es viele Quellenverweise gibt. 
        Beispiele sind~\cite{Ko05a},~\cite{Ko05b},~\cite{MiGo05},~\cite{TeGo14},~\cite{HuHa07}.
        Es wird dringend empfohlen, Biber oder BibTeX zu verwenden (wie in diesen Beispielen).
        
        Siehe dazu auch den Ort wo diese definiert werden \verb|bibliography.bib|, (Templates verfügbar).

        \subsubsection{Code}
        \begin{lstlisting}[language=C++,name={1. Beispiel},label={sc:bsp:1}]
            #include <iostream>

            void SayHello(void)
            {
                // Kommentar
                cout << "Hello World!" << endl;
            }

            int main(int argc, char **argv)
            {
                SayHello();
                return 0;
            }
            \end{lstlisting}

\clearpage